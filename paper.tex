%% For double-blind review submission, w/o CCS and ACM Reference (max submission space)
\documentclass[sigops,10pt,review,anonymous]{acmart}\settopmatter{printfolios=true,printccs=false,printacmref=false}
%% For double-blind review submission, w/ CCS and ACM Reference
%\documentclass[sigplan,review,anonymous]{acmart}\settopmatter{printfolios=true}
%% For single-blind review submission, w/o CCS and ACM Reference (max submission space)
%\documentclass[sigplan,review]{acmart}\settopmatter{printfolios=true,printccs=false,printacmref=false}
%% For single-blind review submission, w/ CCS and ACM Reference
%\documentclass[sigplan,review]{acmart}\settopmatter{printfolios=true}
%% For final camera-ready submission, w/ required CCS and ACM Reference
%\documentclass[sigplan]{acmart}\settopmatter{}

% Packages.
\usepackage{graphicx}
\usepackage{textcomp}
\usepackage{dblfloatfix}

%% Conference information
%% Supplied to authors by publisher for camera-ready submission;
%% use defaults for review submission.
\acmConference[SOSP'2019]{The 27th ACM Symposium on Operating Systems Principles}{October 27-30, 2019}{Huntsville, Ontario, Canada}
\acmYear{2019}
\acmISBN{} % \acmISBN{978-x-xxxx-xxxx-x/YY/MM}
\acmDOI{} % \acmDOI{10.1145/nnnnnnn.nnnnnnn}
\startPage{1}

%% Copyright information
%% Supplied to authors (based on authors' rights management selection;
%% see authors.acm.org) by publisher for camera-ready submission;
%% use 'none' for review submission.
\setcopyright{none}
%\setcopyright{acmcopyright}
%\setcopyright{acmlicensed}
%\setcopyright{rightsretained}
%\copyrightyear{2018}           %% If different from \acmYear

%% Bibliography style
\bibliographystyle{ACM-Reference-Format}
%% Citation style
%\citestyle{acmauthoryear}  %% For author/year citations
%\citestyle{acmnumeric}     %% For numeric citations
%\setcitestyle{nosort}      %% With 'acmnumeric', to disable automatic
                            %% sorting of references within a single citation;
                            %% e.g., \cite{Smith99,Carpenter05,Baker12}
                            %% rendered as [14,5,2] rather than [2,5,14].
%\setcitesyle{nocompress}   %% With 'acmnumeric', to disable automatic
                            %% compression of sequential references within a
                            %% single citation;
                            %% e.g., \cite{Baker12,Baker14,Baker16}
                            %% rendered as [2,3,4] rather than [2-4].


%%%%%%%%%%%%%%%%%%%%%%%%%%%%%%%%%%%%%%%%%%%%%%%%%%%%%%%%%%%%%%%%%%%%%%
%% Note: Authors migrating a paper from traditional SIGPLAN
%% proceedings format to PACMPL format must update the
%% '\documentclass' and topmatter commands above; see
%% 'acmart-pacmpl-template.tex'.
%%%%%%%%%%%%%%%%%%%%%%%%%%%%%%%%%%%%%%%%%%%%%%%%%%%%%%%%%%%%%%%%%%%%%%


%% Some recommended packages.
\usepackage{booktabs}   %% For formal tables:
                        %% http://ctan.org/pkg/booktabs
\usepackage{subcaption} %% For complex figures with subfigures/subcaptions
                        %% http://ctan.org/pkg/subcaption


\begin{document}

%% Title information
\title[]{Concurrent Priority Queues}        %% [Short Title] is optional;
                                        %% when present, will be used in
                                        %% header instead of Full Title.
%\titlenote{with title note}             %% \titlenote is optional;
                                        %% can be repeated if necessary;
                                        %% contents suppressed with 'anonymous'
\subtitle{Setting our priorities straight}                     %% \subtitle is optional
%\subtitlenote{with subtitle note}       %% \subtitlenote is optional;
                                        %% can be repeated if necessary;
                                        %% contents suppressed with 'anonymous'


%% Author information
%% Contents and number of authors suppressed with 'anonymous'.
%% Each author should be introduced by \author, followed by
%% \authornote (optional), \orcid (optional), \affiliation, and
%% \email.
%% An author may have multiple affiliations and/or emails; repeat the
%% appropriate command.
%% Many elements are not rendered, but should be provided for metadata
%% extraction tools.

%% Author with single affiliation.
\author{First1 Last1}
\authornote{with author1 note}          %% \authornote is optional;
                                        %% can be repeated if necessary
\orcid{nnnn-nnnn-nnnn-nnnn}             %% \orcid is optional
\affiliation{
  \position{Position1}
  \department{Department1}              %% \department is recommended
  \institution{Institution1}            %% \institution is required
  \streetaddress{Street1 Address1}
  \city{City1}
  \state{State1}
  \postcode{Post-Code1}
  \country{Country1}                    %% \country is recommended
}
\email{first1.last1@inst1.edu}          %% \email is recommended

%% Author with two affiliations and emails.
\author{First2 Last2}
\authornote{with author2 note}          %% \authornote is optional;
                                        %% can be repeated if necessary
\orcid{nnnn-nnnn-nnnn-nnnn}             %% \orcid is optional
\affiliation{
  \position{Position2a}
  \department{Department2a}             %% \department is recommended
  \institution{Institution2a}           %% \institution is required
  \streetaddress{Street2a Address2a}
  \city{City2a}
  \state{State2a}
  \postcode{Post-Code2a}
  \country{Country2a}                   %% \country is recommended
}
\email{first2.last2@inst2a.com}         %% \email is recommended
\affiliation{
  \position{Position2b}
  \department{Department2b}             %% \department is recommended
  \institution{Institution2b}           %% \institution is required
  \streetaddress{Street3b Address2b}
  \city{City2b}
  \state{State2b}
  \postcode{Post-Code2b}
  \country{Country2b}                   %% \country is recommended
}
\email{first2.last2@inst2b.org}         %% \email is recommended


%% Abstract
%% Note: \begin{abstract}...\end{abstract} environment must come
%% before \maketitle command
\begin{abstract}
The area of concurrent priority queues suffers two problems each of which we aim to alleviate in this paper. The first is an apparent split in the research community with two separate factions each representing their priority queues. These two sub communities, knowingly or not, have benchmarked against disjoint  groups of priority queues and don't reference each other in their respective work. The second is success, with a large number of new concurrent priority queues being published the task of a new researcher looking to find the state of the art has a hard time ahead of them. This problem is compounded by the apparent split in benchmarking and analysis with a researcher potentially landing in one of the two islands and never seeing the other. Our paper solves both of these issues with a comprehensive thorough analysis of concurrent priority queues, an open source implementation of each queue and the benchmarking framework using the DEF language, and a detailed topology of each priority queue so that a new researcher can quickly get up to speed on the area. This paper also attempts to make suggestions for particular use cases dependent on the task at hand for the practitioner reading this work.
\end{abstract}


%% 2012 ACM Computing Classification System (CSS) concepts
%% Generate at 'http://dl.acm.org/ccs/ccs.cfm'.
\begin{CCSXML}
<ccs2012>
<concept>
<concept_id>10010147.10010169.10010175</concept_id>
<concept_desc>Computing methodologies~Parallel programming languages</concept_desc>
<concept_significance>500</concept_significance>
</concept>
</ccs2012>
\end{CCSXML}

\ccsdesc[500]{Computing methodologies~Parallel programming languages}
%% End of generated code


%% Keywords
%% comma separated list
\keywords{multiprocessor, high performance, concurrency, priority queues}  %% \keywords are mandatory in final camera-ready submission


%% \maketitle
%% Note: \maketitle command must come after title commands, author
%% commands, abstract environment, Computing Classification System
%% environment and commands, and keywords command.
\maketitle


\section{Introduction}

We think there are so many priority queues. They also exist in two universes, no one benches CBPQ against the SprayList.

\section{Background}

Broad history of concurrent priority queues.

\section{Implementation}

\subsection{Concurrent Data Structures}


In our paper we attempt to define logical categories that each priority queue could be put into. Initially there are some obvious choices the difference between those which are linearizable and those which are not, the underlying data structure, and/or the underlying technique involved.

\subsection{Skip List Based Priority Queues}

\paragraph{Shavit Lotan Queue} The Shavit Lotan priority queue was the first of a number of skip list-based priority queues. \cite{ShavitLotanQueue} The Shavit Lotan Queue is a lock-free priority queue data-structure where the \texttt{pop-min} method traverses the lowest height/level of the skip list. Since the skip list is a series of sorted linked lists, with the lowest height/level containing all values, threads can remove the top items from the list which have the highest priority. Deletion is done logically, using the atomic swap (or similar CAS) to mark a node as deleted and then physically removing that node at that point or sometime after. The underlying skip list can be lock-free or lock-based, indeed the initial publication used a lock-based skip-list. The algorithm is quiescently consistent as other nodes with a lower value may be skipped over for nodes with a higher value inserted afterwards. However timestamps can be added to each node in order to make the list linearizable \cite{Linearizability}.

\paragraph{Lind{\'e}n Jonsson Queue} The Lind{\'e}n Jonsson Queue \cite{LindenPriority} is a lock-free linearizable priority queue where the \texttt{pop-min} method traverses the lowest height/level of the skip list. The novelty of the Lind{\'e}n Jonsson is a sizable reduction in the number of atomic operations with a new representation of deleted nodes in a skip list and batch deletion. Deleted nodes are not represented by a boolean or a marked next pointer but rather a marked predecessor (node to the left) pointer. The marking is done with an atomic-or operation and is generally considered more efficient than a standard compare and swap. The authors also have a novel batch deletion method for logically deleted nodes piling up at the front of the queue. Batch deletion gives significant performance benefits over traditional priority queues. The length of the logically deleted prefix is given as a tunable parameter, the optimal value of which depends on the workload and the number of processes acting on the priority queue.

\paragraph{SprayList} The last pure skip list based priority queue is the SprayList. The primary contribution of the SprayList is deletion via a probabilistic random walk from the beginning of the list in order to reduce and spread contention. This probabilistic walk is referred to as the \textit{spray} and is paramount to reducing contention from threads. The SprayList is a relaxed concurrent data-structure whereby \texttt{pop-min} will remove from a range of top items in the list. The range and other parameters of the \textit{spray} are initialized by the number of expected processes (concurrency) acting on the queue. The implementation of the \textit{spray} means nodes closer to the beginning are less likely to be \textit{landed-on}, so padding nodes are added to the beginning of the list. Once the SprayList \textit{sprays} to find a node it will continue in the same fashion as the Shavit Lotan Queue and traverse the lowest level of the list until a not deleted node is found. Under high contention the SprayList can scale significantly better than traditional concurrent priority queues, but it can be very sensitive to the choice of spray parameters.

\subsection{Heap/Array Based Priority Queues}

\paragraph{Hunt Heap} Heaps make for extremely efficient serial priority queues having only recently been dethroned by B-Trees.

\paragraph{Mounds} Mounds are a randomized array based 


\section{Experimental Results}

\subsection{Test Configuration}

Our benchmark was tested on a 72-core (4 sockets with 18 cores each capable of running 2 hardware threads, totalling 144 hardware threads) Intel Xeon machine clocked at 2.50 GHz with 512 GB of RAM and 4 memory banks. The machine is running Ubuntu 14.04 with kernel version 3.13.0-141. The with DEF code was compiled with DEF version 0.18.0a and the C code was compiled with Clang 6.0, and all code was compiled with -03 level of optimization.

During the benchmark threads were pinned programmatically to individual cores initially avoiding HyperThreading and later exhausting all individual execution units on a single CPU with the use of HyperThreading, before migrating to another CPU socket. JEMalloc\cite{JEMalloc} was used in all tests, as it had been in the Forkscan paper.\cite{Forkscan} The \texttt{numactl} Linux program was used to control which memory bank allocation was allowed to take place. The memory banks closest to the running CPUs were selected as they became active.

The microbenchmark measures the number of operations carried out over a specified amount of time rather than the time taken to execute a specified number of iterations. The rationale for this is threads finish their iterations before other threads. The remaining threads complete their iterations with less contention in the system, skewing the overall benchmark. Our benchmark reports the number of operations per second. The primary comparison is between the leaky C and the Forkscan enabled DEF implementations. Each benchmark is run for a total of 20 seconds each, with each configuration being sampled 5 times. The average of the runs is plotted.

Lastly, in figure \ref{fig:priorityqueues}, we tested the Lind{\'e}n Jonsson Queue and the SprayList priority queue.  The leaky implementations took a performance hit crossing the NUMA node boundary at 36 cores and never recovered.  One of the difficulties with a skip list-based priority queue is that many threads have a few early nodes in common in their caches.  As thread-counts increase, so do the number of invalidations, leading to the shape of the graph.  This is a worst-case data structure for Forkscan because priority queues are entirely updates, and the nodes are big (180 bytes, in this implementation).  However, again, the retiring DEF implementation has the same shape as the leaky ones.  It scales while they scale, and it's ultimately defeated by the same hurdle. The Lind{\'e}n Jonsson Queue beats the Spray List in our implementation. We found the Spray List's performance to be very sensitive to the parameter choice which scaled differently on for each thread count. One issue is that nodes were physically removed from the Spray List as soon as they were logically marked as deleted, increasing contention and degrading performance. 

In all of these structures, reclaiming memory scales up to the physical limits of the hardware, albeit with a gentler slope.  We attribute this to the on-demand aspect of memory tracking, versus at-allocation tracking; the memory that gets retired can mostly be de-allocated. Very little has to be recursively searched.



\section{Conclusion}

We've shown what is what!

\subsection{Future Work}

None, we've done it all! Probably cache aware objects though...

%% Acknowledgments
\begin{acks}                            %% acks environment is optional
                                        %% contents suppressed with 'anonymous'
  %% Commands \grantsponsor{<sponsorID>}{<name>}{<url>} and
  %% \grantnum[<url>]{<sponsorID>}{<number>} should be used to
  %% acknowledge financial support and will be used by metadata
  %% extraction tools.
  This material is based upon work supported by the
  \grantsponsor{GS100000001}{National Science
    Foundation}{http://dx.doi.org/10.13039/100000001} under Grant
  No.~\grantnum{GS100000001}{nnnnnnn} and Grant
  No.~\grantnum{GS100000001}{mmmmmmm}.  Any opinions, findings, and
  conclusions or recommendations expressed in this material are those
  of the author and do not necessarily reflect the views of the
  National Science Foundation.
\end{acks}


%% Bibliography
\bibliography{bibliography}


%% Appendix
%\appendix
%\section{Appendix}

%Text of appendix \ldots

\end{document}
